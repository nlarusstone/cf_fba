\chapter{Conclusion}
The goal of this dissertation was to provide a metabolic model for \gls{cfps} systems.
I produced a system, \gls{sys}, that accomplished that goal by generate these reduced metabolic models from already existing, genome-scale models.
The models generated by \gls{sys} explain real experimental data better than currently existing models.
The system I created was used to generate \gls{ecoli} \gls{cfps} models, but can be applied to other types of organisms.
Along the way, I developed a new type of loss function for \gls{vae} and showed that it is more effective at discovering patterns in the experimental data than \gls{pca}.

In conclusion, I have shown that deep learning techniques can be successfully applied to biology, specifically metabolic modeling.
Computational biologists should start thinking about using \glspl{ae} (especially \glspl{vae}) for dimensionality reduction instead of defaulting to using \gls{pca}.
In particular, the new loss function that I developed for my \gls{vae} is general enough to be used with any biological data and provides a good starting point for future work.
I hope this work encourages other computational biologists to continue work integrating deep learning with biology and metabolic models.